\chapter{AI and Exfiltration}
\section{The Problem}
Exfiltration is rarely accidental and confidential information does
not move itself. The key question becomes ``How to identify the actor?''

We can take a bottom-up approach to the problem. This involves several
layers, some of which are traditional and some which are not yet in use.

At the network layer we need to have ``sensors'' which can detect the
movement of confidential data across the security boundary. Various
sensing techniques exist. Indeed, we are building a network sensor
prototype which will detect an exfiltration.

At the monitoring layer we need rules that specify what actions to take.
Rule-based systems, usually built on SNORT or Linux IPTables,
allow actions such as DROPing a packet. However, this layer of rules is
too close to the metal and cannot see exfiltration events over normal
traffic avenues like FTP or email.

At the alert level we need network logging, again usually using things
like IPTable LOG rules.

There are issues that arise at the logging layer and above. 
For any network size, logging is useless if the logs are not reviewed.
But, almost no matter the size of the network, the number of people
available to review the logs is not sufficient to act on each event.

A network attack can easly swamp a network log file so that it either
contains too many entries or simply runs out of storage.

Worse, even if every event was tracked, there needs to be someone who
can provide an overview of all the events to extract the common elements.
These common elements need to be ``backtracked'' to find the actor(s).

This ``analysis'' layer is usually missing from network security.
At best it occurs as a post-mortem after a network breach. But given
the volume of traffic locally or in a data center, the likelyhood of
finding an exfiltration pattern is small.

\section{An Artificial Intelligence Approach}
Traditional Artificial Intelligence (AI) techniques can be applied to this
problem. Large volumes of data containing small patterns of information
are the normal domain of these systems. 

Usually the ``sensors'' are human-like, such as cameras for ``eyes'',
microphones for ``ears'', and robot arms for ``actuators''. But that
does not have to be the case. In a network environment the ``sensing''
equipment are low-level hardware and SNORT/IPTables firewall rules.

The AI problem, given the logs and active alerts on the sensors, is to
find elfiltration events, correlate the events with patterns, and from
those patterns identify the likely actor(s).


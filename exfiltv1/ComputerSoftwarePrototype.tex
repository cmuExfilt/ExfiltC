\chapter{Computer Software Prototype}

The goal is to set up two laptops. One is configured as a THREAT
machine, the second is configured as the EXFILT machine.

To configure the user machine we need to install Ubuntu,
hardwire the THREAT machine to the EXFILT machine with a
crossover cable, and set up a lan between the two machines.

For the THREAT machine, the steps are:
\begin{enumerate}
\item Download and burn Ubuntu CD
\item Install Ubuntu from the CD
\item Connect the crossover cable
\item set hostname to THREAT
\begin{enumerate}
\item edit /etc/hostname
\item edit /etc/hosts
\begin{itemize}
\item 127.0.0.1 localhost
\item 127.0.1.1 THREAT
\item 10.0.0.2  THREAT
\end{itemize}
\item edit /etc/network/interfaces
\begin{verbatim}
auto eth0
iface eth0 inet static
address 10.0.0.2
gateway 10.0.0.1
netmask 255.255.255.0
broadcast 10.0.0.255
\end{verbatim}
\end{enumerate}
\item turn on manual network management 
\begin{itemize}
\item edit /etc/NetworkManager/NetworkManger.conf
\begin{itemize}
\item comment out dns by putting \# as first character of the line
\item set managed=true  (says WE are managing the connection)
\end{itemize}
\end{itemize}
\item Set up the lan
\begin{enumerate}
\item sudo ifconfig eth0 10.0.0.2 netmask 255.255.255.0 up
\item sudo route add default gw 10.0.0.1
\end{enumerate}
\end{enumerate}
Notice that the THREAT machine has a single IP address of 10.0.0.2
and routes traffic to 10.0.0.1

SNORT Malware

https://github.com/rshipp/awesome-malware-analysis

For the EXFILT machine, the steps are:
\begin{enumerate}
\item Download and burn Ubuntu CD
\item Install Ubuntu from the CD
\item Set up full screen in virtualbox
\item set hostname to EXFILT
\item Connect the crossover cable
\item Set up the lan
\begin{enumerate}
\item sudo ifconfig eth1 10.0.0.1 netmask 255.255.255.0 up
\item sudo route add default gw 192.168.1.1
\end{enumerate}
\item wireshark
\begin{enumerate}
\item download source https://www.wireshark.org/download.html
\item bunzip, untar
\item apt-get update
\item apt-get install -y bison flex g++ build-essential
\item install Qt
\begin{enumerate}
\item wget http://download.qt.io/official\_releases/qt/5.0/5.0.2/qt-linux-opensource-5.0.2-x86-offline.run
\item chmod +x qt-linux-opensource-5.0.2-x86-offline.run
\item ./qt-linux-opensource-5.0.2-x86-offline.run
\end{enumerate}
\item ./configure
\end{enumerate}
\end{enumerate}
Notice that the EXFILT machine has 2 IP addresses. The first
address is 10.0.0.1 which is on the crossover lan. But EXFILT
also has a wireless address on the 192.168.1 subnet (WLAN1)

With this setup all traffic from the THREAT machine is routed
through the EXFILT machine. We need to set up EXFILT to act
as the router on the 10.0.0 subnet. 

\subsection{EXFILT Router setup}
\begin{itemize}
\item eth1 lan crossover network 10.0.0.0/8
\item wlan1 wireless outside network 192.168.1.0/8
\item EXFILT = 10.0.0.1, THREAT = 10.0.0.2
\end{itemize}

\subsection{EXFILT software setup}
\begin{verbatim}
sudo apt-get install -y flex bison libpcap-dev libpcre3 libpcre3-dev libdnet

tcpdump-4.7.4.tar.gz (http://www.tcpdump.org)
tar -zxf tcpdump-4.7.4.tar.gz
cd tcpdump-4.7.4
./configure && make && sudo make install

wget https://www.snort.org/downloads/snort/daq-2.0.4.tar.gz
tar -zxf daq-2.0.4.tar.gz
cd daq-2.0.4
./configure && make && sudo make install

http://code.google.com/p/libdnet
tar -zxf libdnet-1.12.tgz
cd libdnet-1.12
./configure && make && sudo make install

wget https://www.snort.org/downloads/snort/snort-2.9.7.2.tar.gz
tar -zxf snort-2.9.7.2.tar.gz
cd snort-2.9.7.2
./configure --enable-sourcefire && make && sudo make install

sudo cp /usr/local/lib/libdnet.1.0.1 /usr/local/lib/libdnet.so.1.0.1
sudo /sbin/ldconfig
sudo updatedb
snort -v -i wlan1

\end{verbatim}



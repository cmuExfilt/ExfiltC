\chapter{EXFILT Plan Steps}
EXFILT will begin as a software-based set of technologies to prevent
sensitive information from traveling outside of a computer network.
The intent of the initial version is to create Intellectual Property
and a prototypical Proof of Concept to be used for fundraising.  As
such, hardware/firmware based implementations involving FPGA’s,
microprocessors and other embedded logic controllers are out of scope;
but might be developed at a later date.  Extensibility and
interoperability with potential future non-software based
implementations are not of major concern when building the initial
prototype version.

\vspace{2mm}
\noindent
{\bf Version 0.0}

The goal for this iteration is to construct a SNORT plugin (in C) for
use with Wireshark.  The plugin will use deep packet inspection to
detect a transfer of sensitive information by looking for the presence
of a SHA1 hash.  To allow us to focus on the complexities of getting a
basic plugin functioning, the only protocol and transfer type
addressed by this version will be an FTP Copy operation; and test data
will simply be unencrypted plain-text files.  Other formats and modes
like .docx and .pdf files, or email attachments will be addressed in
later iterations.

When the plugin detects sensitive information it will log the
date/time, origin IP, destination IP, and other details depending on
availability and usefulness.  At this stage it will not attempt to
terminate or block the transmission of data.

\vspace{2mm}
\noindent
{\bf Version 0.1}

This iteration involves the construction of a proxy server-like buffer
(in Java) for accumulating the entire set of packets in a
transmission, for subsequent inspection of file attributes (such as
file type), and syntactic analysis against a cohesive file.  The
buffering system at this point does not need to implement any
filtering/detection methods.  It simply needs to be able to accumulate
the packets in a transmission, then either stop the transmission by
sending it forward, or forward the transmission on to its final
destination.

\vspace{2mm}
\noindent
{\bf Version 0.2}

This iteration of the prototype will include one or more of the
following, depending on priorities, time, resources and discoveries as
they stand at the completion of Version 0.1:

\begin{enumerate}[A.]
\item An interface for end users to define syntactic filters (words,
phrases and patterns) to be used for detecting sensitive information
in transmissions accumulated in the Buffering system built in v0.1

\item Automated redaction of sensitive information through replacement of
detected sensitive information patterns with innocuous text prior to
re-transmission to the final destination

\item Support for file formats other than plain text (e.g.: .docx, .pdf,
.xls, etc…) Open source, Java-based readers for various file formats
might be used for this effort.  Since the goal of Version 0 is a
functional Proof Of Concept, proprietary document readers with lower
latency can be built in later development efforts.

\item Support for filtering email attachments, as well as communication
protocols other than FTP.  This effort could encompass changes to both
the Buffering and Filtering system and the SNORT plugin built in
version 0.0.
\end{enumerate}


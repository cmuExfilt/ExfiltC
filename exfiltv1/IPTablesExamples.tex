\chapter{IP Tables Examples\cite{9}}
\subsection{Show firewall status}

Type the following command as root:

\begin{verbatim}
# iptables -L -n -v
\end{verbatim}

Inactive firewall output:
\begin{verbatim}
Chain INPUT (policy ACCEPT 0 packets, 0 bytes)
 pkts bytes target     prot opt in     out     source               destination
Chain FORWARD (policy ACCEPT 0 packets, 0 bytes)
 pkts bytes target     prot opt in     out     source               destination
Chain OUTPUT (policy ACCEPT 0 packets, 0 bytes)
 pkts bytes target     prot opt in     out     source               destination
\end{verbatim}

active firewall output:
\begin{verbatim}
Chain INPUT (policy DROP 0 packets, 0 bytes)
 pkts bytes target     prot opt in     out     source               destination
    0     0 DROP       all  --  *      *       0.0.0.0/0            0.0.0.0/0 
          state INVALID
  394 43586 ACCEPT     all  --  *      *       0.0.0.0/0            0.0.0.0/0 
          state RELATED,ESTABLISHED
   93 17292 ACCEPT     all  --  br0    *       0.0.0.0/0            0.0.0.0/0
    1   142 ACCEPT     all  --  lo     *       0.0.0.0/0            0.0.0.0/0
Chain FORWARD (policy DROP 0 packets, 0 bytes)
 pkts bytes target     prot opt in     out     source               destination
    0     0 ACCEPT     all  --  br0    br0     0.0.0.0/0            0.0.0.0/0
    0     0 DROP       all  --  *      *       0.0.0.0/0            0.0.0.0/0 
          state INVALID
    0     0 TCPMSS     tcp  --  *      *       0.0.0.0/0            0.0.0.0/0 
          tcp flags:0x06/0x02 TCPMSS clamp to PMTU
    0     0 ACCEPT     all  --  *      *       0.0.0.0/0            0.0.0.0/0 
          state RELATED,ESTABLISHED
    0     0 wanin      all  --  vlan2  *       0.0.0.0/0            0.0.0.0/0
    0     0 wanout     all  --  *      vlan2   0.0.0.0/0            0.0.0.0/0
    0     0 ACCEPT     all  --  br0    *       0.0.0.0/0            0.0.0.0/0
Chain OUTPUT (policy ACCEPT 425 packets, 113K bytes)
 pkts bytes target     prot opt in     out     source               destination
Chain wanin (1 references)
 pkts bytes target     prot opt in     out     source               destination
Chain wanout (1 references)
 pkts bytes target     prot opt in     out     source               destination
\end{verbatim}
Where,
\begin{itemize}
\item -L : List rules.
\item -v : Display detailed information.
\item -n : Display IP address and port in numeric format
\end{itemize}

\subsection{Firewall with line numbers}
\begin{verbatim}
# iptables -n -L -v --line-numbers
\end{verbatim}

Sample outputs:
\begin{verbatim}
Chain INPUT (policy DROP)
num  target     prot opt source               destination
1    DROP       all  --  0.0.0.0/0            0.0.0.0/0       
    state INVALID
2    ACCEPT     all  --  0.0.0.0/0            0.0.0.0/0       
    state RELATED
,ESTABLISHED
3    ACCEPT     all  --  0.0.0.0/0            0.0.0.0/0
4    ACCEPT     all  --  0.0.0.0/0            0.0.0.0/0
Chain FORWARD (policy DROP)
num  target     prot opt source               destination
1    ACCEPT     all  --  0.0.0.0/0            0.0.0.0/0
2    DROP       all  --  0.0.0.0/0            0.0.0.0/0       
    state INVALID
3    TCPMSS     tcp  --  0.0.0.0/0            0.0.0.0/0           
    tcp flags:0x06/0x02 TCPMSS clamp to PMTU
4    ACCEPT     all  --  0.0.0.0/0            0.0.0.0/0       
    state RELATED,ESTABLISHED
5    wanin      all  --  0.0.0.0/0            0.0.0.0/0
6    wanout     all  --  0.0.0.0/0            0.0.0.0/0
7    ACCEPT     all  --  0.0.0.0/0            0.0.0.0/0
Chain OUTPUT (policy ACCEPT)
num  target     prot opt source               destination
Chain wanin (1 references)
num  target     prot opt source               destination
Chain wanout (1 references)
num  target     prot opt source               destination
\end{verbatim}

You can use line numbers to delete or insert new rules into the firewall.

\subsection{INPUT or OUTPUT chain rules}
\begin{verbatim}
# iptables -L INPUT -n -v
# iptables -L OUTPUT -n -v --line-numbers
\end{verbatim}

\subsection{Stop / Start / Restart the Firewall}
If you are using CentOS / RHEL / Fedora Linux, enter:
\begin{verbatim}
# service iptables stop
# service iptables start
# service iptables restart
\end{verbatim}

You can use the iptables command itself to stop the firewall 
and delete all rules:
\begin{verbatim}
# iptables -F
# iptables -X
# iptables -t nat -F
# iptables -t nat -X
# iptables -t mangle -F
# iptables -t mangle -X
# iptables -P INPUT ACCEPT
# iptables -P OUTPUT ACCEPT
# iptables -P FORWARD ACCEPT
\end{verbatim}

Where,
\begin{itemize}
\item -F : Delete all the rules.
\item -X : Delete chain.
\item -t table\_name : Select table (called nat or mangle) 
and delete/flush rules.
\item -P : Set the default policy.
\end{itemize}

\subsection{Delete Firewall Rules}

To display line number along with other information for existing rules, enter:
\begin{verbatim}
# iptables -L INPUT -n --line-numbers
# iptables -L OUTPUT -n --line-numbers
# iptables -L OUTPUT -n --line-numbers | less
# iptables -L OUTPUT -n --line-numbers | grep 202.54.1.1
\end{verbatim}

You will get the list of IP. Look at the number on the left, then use 
number to delete it. For example delete line number 4, enter:
\begin{verbatim}
# iptables -D INPUT 4
\end{verbatim}

OR find source IP 202.54.1.1 and delete from rule:
\begin{verbatim}
# iptables -D INPUT -s 202.54.1.1 -j DROP
\end{verbatim}
Where,
\begin{itemize}
\item -D : Delete one or more rules from the selected chain
\end{itemize}

\subsection{Insert Firewall Rules}

To insert one or more rules in the selected chain as the given rule 
number use the following syntax. First find out line numbers, enter:
\begin{verbatim}
# iptables -L INPUT -n –line-numbers
\end{verbatim}

Sample outputs:
\begin{verbatim}
Chain INPUT (policy DROP)
num  target     prot opt source               destination
1    DROP       all  --  202.54.1.1           0.0.0.0/0
2    ACCEPT     all  --  0.0.0.0/0            0.0.0.0/0      
     state NEW,ESTABLISHED
\end{verbatim}

To insert rule between 1 and 2, enter:
\begin{verbatim}
# iptables -I INPUT 2 -s 202.54.1.2 -j DROP
\end{verbatim}

To view updated rules, enter:
\begin{verbatim}
# iptables -L INPUT -n --line-numbers
\end{verbatim}

Sample outputs:
\begin{verbatim}
Chain INPUT (policy DROP)
num  target     prot opt source               destination
1    DROP       all  --  202.54.1.1           0.0.0.0/0
2    DROP       all  --  202.54.1.2           0.0.0.0/0
3    ACCEPT     all  --  0.0.0.0/0            0.0.0.0/0      
     state NEW,ESTABLISHED
\end{verbatim}

\subsection{Save Firewall Rules}

To save firewall rules under CentOS / RHEL / Fedora Linux, enter:
\begin{verbatim}
# service iptables save
\end{verbatim}

In this example, drop an IP and save firewall rules:
\begin{verbatim}
# iptables -A INPUT -s 202.5.4.1 -j DROP
# service iptables save
\end{verbatim}

For all other distros use the iptables-save command:
\begin{verbatim}
# iptables-save > /root/my.active.firewall.rules
# cat /root/my.active.firewall.rules
\end{verbatim}

\subsection{Restore Firewall Rules}

To restore firewall rules form a file called 
/root/my.active.firewall.rules, enter:
\begin{verbatim}
# iptables-restore < /root/my.active.firewall.rules
\end{verbatim}

To restore firewall rules under CentOS / RHEL / Fedora Linux, enter:
\begin{verbatim}
# service iptables restart
\end{verbatim}

\subsection{Set the Default Firewall Policies}

To drop all traffic:
\begin{verbatim}
# iptables -P INPUT DROP
# iptables -P OUTPUT DROP
# iptables -P FORWARD DROP
# iptables -L -v -n
\end{verbatim}

NOTE: You will not able to connect anywhere as all traffic is dropped
\begin{verbatim}
# ping teknixx.com
# wget http://www.kernel.org/pub/linux/kernel/v3.0/testing/linux-3.2-rc5.tar.bz2
\end{verbatim}

\subsection{Only Block Incoming Traffic}

To drop all incoming / forwarded packets, but allow outgoing traffic, enter:
\begin{verbatim}
# iptables -P INPUT DROP
# iptables -P FORWARD DROP
# iptables -P OUTPUT ACCEPT
# iptables -A INPUT -m state --state RELATED,ESTABLISHED -j ACCEPT
# iptables -L -v -n
\end{verbatim}

Now ping and wget should work
\begin{verbatim}
# ping teknixx.com
# wget http://www.kernel.org/pub/linux/kernel/v3.0/testing/linux-3.2-rc5.tar.bz2
\end{verbatim}

\subsection{Drop Private Network Address On Public Interface}

IP spoofing is nothing but to stop the following IPv4 address ranges
for private networks on your public interfaces. Packets with
non-routable source addresses should be rejected using the following
syntax:
\begin{verbatim}
# iptables -A INPUT -i eth1 -s 192.168.0.0/24 -j DROP
# iptables -A INPUT -i eth1 -s 10.0.0.0/8 -j DROP
\end{verbatim}

IPv4 Address Ranges For Private Networks
\begin{itemize}
\item 10.0.0.0/8 -j (A)
\item 172.16.0.0/12 (B)
\item 192.168.0.0/16 (C)
\item 224.0.0.0/4 (MULTICAST D)
\item 240.0.0.0/5 (E)
\item 127.0.0.0/8 (LOOPBACK)
\end{itemize}

\subsection{Blocking an IP Address}

To block an attackers ip address called 1.2.3.4, enter:
\begin{verbatim}
# iptables -A INPUT -s 1.2.3.4 -j DROP
# iptables -A INPUT -s 192.168.0.0/24 -j DROP
\end{verbatim}

\subsection{Block Incoming Port}

To block all service requests on port 80, enter:
\begin{verbatim}
# iptables -A INPUT -p tcp --dport 80 -j DROP
# iptables -A INPUT -i eth1 -p tcp --dport 80 -j DROP
\end{verbatim}

To block port 80 only for an ip address 1.2.3.4, enter:
\begin{verbatim}
# iptables -A INPUT -p tcp -s 1.2.3.4 --dport 80 -j DROP
# iptables -A INPUT -i eth1 -p tcp -s 192.168.1.0/24 --dport 80 -j DROP
\end{verbatim}

\subsection{Block Outgoing IP Address}

To block outgoing traffic to a particular host or domain 
such as teknixx.com, enter:
\begin{verbatim}
# host -t a teknixx.com
\end{verbatim}

Sample outputs:
teknixx.com has address 75.126.153.206
Note down its ip address and type the following to block 
all outgoing traffic to 75.126.153.206:
\begin{verbatim}
# iptables -A OUTPUT -d 75.126.153.206 -j DROP
\end{verbatim}

You can use a subnet as follows:
\begin{verbatim}
# iptables -A OUTPUT -d 192.168.1.0/24 -j DROP
# iptables -A OUTPUT -o eth1 -d 192.168.1.0/24 -j DROP
\end{verbatim}

\subsection{Block Domain}

First, find out all ip address of facebook.com, enter:
\begin{verbatim}
# host -t a www.facebook.com
\end{verbatim}

Sample outputs:
\begin{verbatim}
www.facebook.com has address 69.171.228.40
\end{verbatim}

Find CIDR for 69.171.228.40, enter:
\begin{verbatim}
# whois 69.171.228.40 | grep CIDR
\end{verbatim}

Sample outputs:
\begin{verbatim}
CIDR:           69.171.224.0/19
\end{verbatim}

To prevent outgoing access to www.facebook.com, enter:
\begin{verbatim}
# iptables -A OUTPUT -p tcp -d 69.171.224.0/19 -j DROP
\end{verbatim}

You can also use domain name, enter:
\begin{verbatim}
# iptables -A OUTPUT -p tcp -d www.facebook.com -j DROP
# iptables -A OUTPUT -p tcp -d facebook.com -j DROP
\end{verbatim}

From the iptables man page:

… specifying any name to be resolved with a remote query such as DNS
(e.g., facebook.com is a really bad idea), a network IP address (with
/mask), or a plain IP address …

\subsection{Log and Drop Packets}

Type the following to log and block IP spoofing on public interface called eth1
\begin{verbatim}
# iptables -A INPUT -i eth1 -s 10.0.0.0/8 -j LOG --log-prefix "IP_SPOOF A: "
# iptables -A INPUT -i eth1 -s 10.0.0.0/8 -j DROP
\end{verbatim}

By default everything is logged to /var/log/messages file.
\begin{verbatim}
# tail -f /var/log/messages
# grep --color 'IP SPOOF' /var/log/messages
\end{verbatim}

\subsection{Log and Drop Packets}

The -m limit module can limit the number of log entries created per
time. This is used to prevent flooding your log file. To log and drop
spoofing per 5 minutes, in bursts of at most 7 entries .
\begin{verbatim}
# iptables -A INPUT -i eth1 -s 10.0.0.0/8 -m limit --limit 5/m --limit-burst 7 -j LOG --log-prefix "IP_SPOOF A: "
# iptables -A INPUT -i eth1 -s 10.0.0.0/8 -j DROP
\end{verbatim}

\subsection{Drop or Accept Traffic From Mac Address}

Use the following syntax:
\begin{verbatim}
# iptables -A INPUT -m mac --mac-source 00:0F:EA:91:04:08 -j DROP
## *only accept traffic for TCP port # 8080 from mac 00:0F:EA:91:04:07 * ##
# iptables -A INPUT -p tcp --destination-port 22 -m mac --mac-source 00:0F:EA:91:04:07 -j ACCEPT
\end{verbatim}

\subsection{Block or Allow Ping Request}

Type the following command to block ICMP ping requests:
\begin{verbatim}
# iptables -A INPUT -p icmp --icmp-type echo-request -j DROP
# iptables -A INPUT -i eth1 -p icmp --icmp-type echo-request -j DROP
\end{verbatim}

Ping responses can also be limited to certain networks or hosts:
\begin{verbatim}
# iptables -A INPUT -s 192.168.1.0/24 -p icmp --icmp-type echo-request -j ACCEPT
\end{verbatim}

The following only accepts limited type of ICMP requests:
\begin{verbatim}
### ** assumed that default INPUT policy set to DROP ** #############
iptables -A INPUT -p icmp --icmp-type echo-reply -j ACCEPT
iptables -A INPUT -p icmp --icmp-type destination-unreachable -j ACCEPT
iptables -A INPUT -p icmp --icmp-type time-exceeded -j ACCEPT
## ** all our server to respond to pings ** ##
iptables -A INPUT -p icmp --icmp-type echo-request -j ACCEPT
\end{verbatim}

\subsection{Open Range of Ports}

Use the following syntax to open a range of ports:
\begin{verbatim}
iptables -A INPUT -m state --state NEW -m tcp -p tcp --dport 7000:7010 -j ACCEPT
\end{verbatim}

\subsection{Open Range of IP Addresses}

Use the following syntax to open a range of IP address:
\begin{verbatim}
## only accept connection to tcp port 80 (Apache) 
## if ip is between 192.168.1.100 and 192.168.1.200 
iptables -A INPUT -p tcp --destination-port 80 -m iprange --src-range 192.168.1.100-192.168.1.200 -j ACCEPT
## nat example ##
iptables -t nat -A POSTROUTING -j SNAT --to-source 192.168.1.20-192.168.1.25
\end{verbatim}

\subsection{Established Connections and Restaring The Firewall}

When you restart the iptables service it will drop established
connections as it unload modules from the system under RHEL / Fedora /
CentOS Linux. Edit, /etc/sysconfig/iptables-config and set
IPTABLES\_MODULES\_UNLOAD as follows:
\begin{verbatim}
IPTABLES_MODULES_UNLOAD = no
\end{verbatim}

\subsection{Help Iptables Flooding My Server Screen}

Use the crit log level to send messages to a log file instead of console:

\begin{verbatim}
iptables -A INPUT -s 1.2.3.4 -p tcp --destination-port 80 -j LOG --log-level crit
\end{verbatim}

\subsection{Block or Open Common Ports}

The following shows syntax for opening and closing common TCP and UDP ports:
 
\begin{verbatim}
Replace ACCEPT with DROP to block port:
## open port ssh tcp port 22 ##
iptables -A INPUT -m state --state NEW -m tcp -p tcp --dport 22 -j ACCEPT
iptables -A INPUT -s 192.168.1.0/24 -m state --state NEW -p tcp --dport 22 -j ACCEPT
 
## open cups (printing service) udp/tcp port 631 for LAN users ##
iptables -A INPUT -s 192.168.1.0/24 -p udp -m udp --dport 631 -j ACCEPT
iptables -A INPUT -s 192.168.1.0/24 -p tcp -m tcp --dport 631 -j ACCEPT
 
## allow time sync via NTP for lan users (open udp port 123) ##
iptables -A INPUT -s 192.168.1.0/24 -m state --state NEW -p udp --dport 123 -j ACCEPT
 
## open tcp port 25 (smtp) for all ##
iptables -A INPUT -m state --state NEW -p tcp --dport 25 -j ACCEPT
 
# open dns server ports for all ##
iptables -A INPUT -m state --state NEW -p udp --dport 53 -j ACCEPT
iptables -A INPUT -m state --state NEW -p tcp --dport 53 -j ACCEPT
 
## open http/https (Apache) server port to all ##
iptables -A INPUT -m state --state NEW -p tcp --dport 80 -j ACCEPT
iptables -A INPUT -m state --state NEW -p tcp --dport 443 -j ACCEPT
 
## open tcp port 110 (pop3) for all ##
iptables -A INPUT -m state --state NEW -p tcp --dport 110 -j ACCEPT
 
## open tcp port 143 (imap) for all ##
iptables -A INPUT -m state --state NEW -p tcp --dport 143 -j ACCEPT
 
## open access to Samba file server for lan users only ##
iptables -A INPUT -s 192.168.1.0/24 -m state --state NEW -p tcp --dport 137 -j ACCEPT
iptables -A INPUT -s 192.168.1.0/24 -m state --state NEW -p tcp --dport 138 -j ACCEPT
iptables -A INPUT -s 192.168.1.0/24 -m state --state NEW -p tcp --dport 139 -j ACCEPT
iptables -A INPUT -s 192.168.1.0/24 -m state --state NEW -p tcp --dport 445 -j ACCEPT
 
## open access to proxy server for lan users only ##
iptables -A INPUT -s 192.168.1.0/24 -m state --state NEW -p tcp --dport 3128 -j ACCEPT
 
## open access to mysql server for lan users only ##
iptables -I INPUT -p tcp --dport 3306 -j ACCEPT
\end{verbatim}

\subsection{Restrict the number of parallel connections}

You can use connlimit module to put such restrictions. To allow 3 ssh connections per client host, enter:
\begin{verbatim}
# iptables -A INPUT -p tcp --syn --dport 22 -m connlimit --connlimit-above 3 -j REJECT
\end{verbatim}

Set HTTP requests to 20:
\begin{verbatim}
# iptables -p tcp --syn --dport 80 -m connlimit --connlimit-above 20 --connlimit-mask 24 -j DROP
\end{verbatim}
Where,
\begin{itemize}
\item –connlimit-above 3 : Match if the number of existing 
connections is above 3.
\item –connlimit-mask 24 : Group hosts using the prefix length. 
For IPv4, this must be a number between (including) 0 and 32.
\end{itemize}


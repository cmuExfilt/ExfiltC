\chapter{Preface}
Exfiltration of confidential data is a growing concern as collections
of confidential information grow larger, these become high-value
targets. One recent example is the loss of 4.2 million records
containing the background checks for top secret security clearance
were exfiltrated from the Office of Personnel Management.

There are several layers of defense that could have, and should have,
been applied to this data. Apparently there was no mechanism to
recognize that this data was being stolen. At minimum, there ought to
be a last line of defense alert system.

A last line of defense would at least make people aware that 
confidential data was being copied. It should monitor the network
for confidential data transfer and, if any is found, raise an alarm.

The idea behind the current work is to prototype such a last line of
defense. We consider marking confidential documents held in storage
and examining data streams to detect that a confidential document is
being exfiltrated.

In this simple prototype we make several simplifying assumptions.

First, we assume that the data is marked by computing a SHA1 hash.
This hash can be kept in a network device table. The network device
computes a hash of document traffic. If the traffic hash matches a
hash in the table then it is a confidential document and an alert is
issued. The ``marker'' can be anything but SHA1 is simple.

Second, we assume that our device is ``plug-and-play'' in that it can
be plugged into the network just prior to leaving the premise.  It
could be applied in an internal network if there is a secure perimeter
around the data.

Third, we assume that the data in transit is a copy of the data in
storage. That is, there is no encryption or obfuscation applied to the
data between reading from storage and arriving at our device.

Our prototype is originally implemented in C running on a PC in order
to examine issues. However, it is clear that such an implementation
cannot keep up with network traffic.

After completing the first prototype we moved on to a Field
Programmable Gate Array (FPGA) hardware implementation. This can
operate at network speeds. We have a VHDL hardware interface which
feeds a primitive VHDL CPU. The CPU is programmed in Forth, a stack
based language.

\section{Overview}

We set up dual laptops, one containing the confidential information
and one trying to exfiltrate it. We used Wireshark to get packet
traces. We wrote C code to extract the confidential ile on the wire
and recognize that it was confidential. This was the initial prototype
plan.

Once that prototype was done we moved to an FPGA implementation. The
implementation uses a VHDL J1 processor which can be programmed in
Forth. We are working to get the bytes off the wire, into memory, 
and checking (using SHA1 in Forth) that we have the confidential file.
This is a hardware implementation of the prototype.



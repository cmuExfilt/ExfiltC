\setlength{\textwidth}{400pt}

%%%%%%%%%%%%%%%%%%%%%%%%%%%%%%%%%%%%%%%%%%%%%%%%%%%%%%%%%%%%%%%%%%%%%%%%%%
%%% Axiom Literate Programming Chunk Support
%%%%%%%%%%%%%%%%%%%%%%%%%%%%%%%%%%%%%%%%%%%%%%%%%%%%%%%%%%%%%%%%%%%%%%%%%%
%% This defines the TeX support for Axiom.

%% Latex Chunk support
%% This is the chunk environment that replaces the use of web-like tools
%%
%% \begin{verbatim}
%% To use the command you would write
%%    \begin{chunk}{some random string}
%%    random code to be verbatim formatted
%%    \end{chunk}
%% 
%%  This version prints 
%%                     --- some random string ---
%%    random code to be verbatim formatted
%%                     --------------------------
%% \end{verbatim}

%%%%%%%%%%%%%%%%%%%%%%%%%%%%%%%%%%%%%%%%%%%%%%%%%%%%%%%%%%%%%%%%%%%%%%%%%%
%%% The verbatim package quotes everything within its grasp and is used to
%%% hide and quote the source code during latex formatting. The verbatim
%%% environment is built in but the package form lets us use it in our
%%% chunk environment and it lets us change the font.
%%%

\usepackage{verbatim}

%%%%%%%%%%%%%%%%%%%%%%%%%%%%%%%%%%%%%%%%%%%%%%%%%%%%%%%%%%%%%%%%%%%%%%%%%%
%%% 
%%% Make the verbatim font smaller
%%% Note that we have to temporarily change the '@' to be just a character
%%% because the \verbatim@font name uses it as a character
%%%

\chardef\atcode=\catcode`\@
\catcode`\@=11
\renewcommand{\verbatim@font}{\ttfamily\small}
\catcode`\@=\atcode

%%%%%%%%%%%%%%%%%%%%%%%%%%%%%%%%%%%%%%%%%%%%%%%%%%%%%%%%%%%%%%%%%%%%%%%%%%
%%% This declares a new environment named ``chunk'' which has one
%%% argument that is the name of the chunk. All code needs to live
%%% between the \begin{chunk}{name} and the \end{chunk}
%%% The ``name'' is used to define the chunk.
%%% Reuse of the same chunk name later concatenates the chunks

%%% For those of you who can't read latex this says:
%%% Make a new environment named chunk with one argument
%%% The first block is the code for the \begin{chunk}{name}
%%% The second block is the code for the \end{chunk}
%%% The % is the latex comment character

%%% We have two alternate markers, a lightweight one using dashes
%%% and a heavyweight one using the \begin and \end syntax
%%% You can choose either one by changing the comment char in column 1
 
\newenvironment{chunk}[1]{%   we need the chunkname as an argument
{\ }\newline\noindent%                    make sure we are in column 1
%{\small $\backslash{}$begin\{chunk\}\{{\bf #1}\}}% alternate begin mark
\hbox{\hskip 2.0cm}{\bf --- #1 ---}%      mark the beginning
\verbatim}%                               say exactly what we see
{\endverbatim%                            process \end{chunk}
\par{}%                                   we add a newline
\noindent{}%                              start in column 1
\hbox{\hskip 2.0cm}{\bf ----------}%      mark the end
%$\backslash{}$end\{chunk\}%              alternate end mark (commented)
\par%                                     and a newline
\normalsize\noindent}%                    and return to the document

%%%%%%%%%%%%%%%%%%%%%%%%%%%%%%%%%%%%%%%%%%%%%%%%%%%%%%%%%%%%%%%%%%%%%%%%%%
%%% This declares the place where we want to expand a chunk

\providecommand{\getchunk}[1]{%
\noindent%
{\small $\backslash{}$begin\{chunk\}\{{\bf #1}\}}}% mark the reference

%%%%%%%%%%%%%%% end the literate program chunking code %%%%%%%%%%%%%%%%%%%

\usepackage[toc,page]{appendix} % add the appendix and a toc entry
\usepackage{multicol}           % split into multiple cols
\usepackage{color}
\definecolor{mygrey}{gray}{0.80}
\usepackage{graphicx}           % for includegraphics
\usepackage{url}                % for format of URLs
\usepackage{enumerate}          % for letters on enumerate
\usepackage{makeidx}            % for constructing the index
\makeindex                      % flag that we construct an index
\usepackage{hyperref}           % for hyperlinking (use blue for links)
\hypersetup{colorlinks=true,linkcolor=blue,pdfborderstyle={/S/U/W 1},
citecolor=red}

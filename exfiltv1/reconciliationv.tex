\chapter{reconciliation}

Here we look into the reconciliation layer, which deals with all the differences between MII and GMII (and eventually RGMII).

\begin{chunk}{inoutputs}

$
module reconciliation
(input clk_125, input reset_n,
 
 // decides whether we are in gmii or mii mode
 input gmii,

 input [7:0] int_tx_dout,
 input int_tx_en,
 input int_tx_er,
 output int_tx_clk,

 output [7:0] int_rx_din,
 output int_rx_dv,
 output int_rx_er,
 output int_rx_clk,
 
 output [7:0] phy_TXD,
 output       phy_TXEN,
 output       phy_TXER,
 output       phy_GTXCLK,
 input        phy_TXCLK,
 input  [7:0] phy_RXD,
 input        phy_RXDV,
 input        phy_RXER,
 input        phy_RXCLK
 );	
$

\end{chunk}

\begin{enumerate}
\item Transmission

Generate $int_tx_clk$, which is what the mac is clocking data in on,
and $tx_clk$ which is what we are clocking data out on. If we are in
MII mode, we ship 4 bits every $phy_TXCLK$, so we want the mac to
clock the data in every other $phy_TXCLK$ since the mac interface is
8 bits.  We divide $phy_TXCLK$ by two and send that as the
$int_tx_clk$.  Otherwise, we are in GMII mode and sending 8 bits
every $phy_GTXCLK$, so send the full $phy_GTXCLK$ as $int_tx_clk$.

\begin{chunk}{transmiss}
$
wire mii_tx_clk;
greg mtc_reg(phy_TXCLK, ~mii_tx_clk, 1'b0, ~reset_n, 1'b1, mii_tx_clk);
BUFGMUX int_tx_clk_mux(.O(int_tx_clk), .I0(mii_tx_clk), .I1(clk_125), .S(gmii));

// Now generate tx_clk, which is used to clock the data out, this is
// simply clk_125 if gmii or phy_TXCLK if not.

wire tx_clk;
// BUFGMUX tx_clk_mux(.O(tx_clk), .I0(phy_TXCLK), .I1(clk_125), .S(gmii));
assign tx_clk = phy_TXCLK;


// Generate the output GTX clock to the phy, this will only be used in
// gmii mode.  We use an ODDR to delay and flip the clock so that
// the rising edge is sent in the middle of the data (sent below)
wire tmp0 = 1'b0;
wire tmp1 = 1'b1;
device_ODDR gtx_clk_out(.S(tmp0), .R(tmp0), .CE(tmp1), 
                        .D0(1'b0), .C0(tx_clk), .D1(1'b1), .C1(~tx_clk), 
                        .Q(phy_GTXCLK));

// now deal with the data.  For mii we alternate between nibbles.  For
// gmii we just blast the whole int_tx_dout each clock.
wire mii_tx_sel;
greg mts_reg(tx_clk, ~mii_tx_sel, ~int_tx_en, ~reset_n, 1'b1, mii_tx_sel);

wire [3:0] mii_txd;
gmux #(4, 1) mii_txd_mux(.d(int_tx_dout), .sel(mii_tx_sel), .z(mii_txd));

wire [7:0] next_txd;
gmux #(8, 1) txd_sel(.d({int_tx_dout, 4'h0, mii_txd}), .sel(gmii), .z(next_txd));

// TODO crossing clock domains... metastability??  Should be in phase...

greg #(8) TXD_reg(tx_clk, next_txd, 1'b0, ~reset_n, 1'b1, phy_TXD);
greg #(1) TXEN_reg(tx_clk, int_tx_en, 1'b0, ~reset_n, 1'b1, phy_TXEN);
greg #(1) TXER_reg(tx_clk, int_tx_er, 1'b0, ~reset_n, 1'b1, phy_TXER);
$
\end{chunk}

\item Reception

Again deal with the clocks first.  If we are in MII mode, the phy
sends 4 bits every $phy_RXCLK$, so the mac should be running at half
$phy_RXCLK$ to get 8 bits each clock tick.  If we are in GMII mode,
the phy sends 8 bits every $phy_RXCLK$, so we can just send the
clock directly through.

\begin{chunk}{receptio}
$
wire mii_rx_clk;
greg rcc_reg(phy_RXCLK, ~mii_rx_clk, 1'b0, ~reset_n, 1'b1, mii_rx_clk);
BUFGMUX int_rx_clk_mux(.O(int_rx_clk), .I0(mii_rx_clk), .I1(phy_RXCLK), .S(gmii));

// the data is a bit harder.  We start by registering the signals
// coming in the from the phy
wire [7:0] rx_din_0;
wire [3:0] rx_din_1;
wire rx_dv_0, rx_dv_1;
wire rx_er_0, rx_er_1;
greg #(8) rx_din_r0(phy_RXCLK, phy_RXD, 1'b0, ~reset_n, 1'b1, rx_din_0);
greg #(1) rx_dv_r0(phy_RXCLK, phy_RXDV, 1'b0, ~reset_n, 1'b1, rx_dv_0);
greg #(1) rx_er_r0(phy_RXCLK, phy_RXER, 1'b0, ~reset_n, 1'b1, rx_er_0);
$
\end{chunk}

the above signal are fine for gmii, but for mii, we need to hold
them for two $phy_RXCLK$ cycles, and not just any two, a cycle with
$int_rx_clk$ low and a cycle with $int_rx_clk$ high. So we want to
bring a new value in when $int_rx_clk$ is high.  To know what delay
registers to use, we sample it when $phy_RXDV$ goes high indicating
a new frame.  If $int_rx_clk$ is high on $new_frame$ is high, $rx_din$
will be get it's first nibble when $int_rx_clk$ is low.


\begin{chunk}{recept2}
$
wire new_frame = ~rx_dv_0 & phy_RXDV;
wire started_on_low;
greg #(1) sol_reg(phy_RXCLK, mii_rx_clk, 1'b0, ~reset_n, new_frame, started_on_low);

greg #(4) rx_din_r1(phy_RXCLK, rx_din_0[3:0], 1'b0, ~reset_n, 1'b1, rx_din_1);
greg #(1) rx_dv_r1(phy_RXCLK, rx_dv_0, 1'b0, ~reset_n, 1'b1, rx_dv_1);
greg #(1) rx_er_r1(phy_RXCLK, rx_er_0, 1'b0, ~reset_n, 1'b1, rx_er_1);

wire [7:0] next_mii_rx_din;
wire next_mii_rx_dv;
wire next_mii_rx_er;
gmux #(10, 1) mii_rx_din_mux
  (.d({phy_RXER, phy_RXDV & ~new_frame, phy_RXD[3:0], rx_din_0[3:0],
       rx_er_1, rx_dv_1, rx_din_0[3:0], rx_din_1[3:0]}),
   .sel(started_on_low),
   .z({next_mii_rx_er, next_mii_rx_dv, next_mii_rx_din}));

wire [7:0] mii_rx_din; 
wire mii_rx_dv;
wire mii_rx_er;
greg #(8) mii_rx_din_r1(phy_RXCLK, next_mii_rx_din, 1'b0, ~reset_n, ~mii_rx_clk, mii_rx_din);
greg #(1) mii_rx_dv_r1(phy_RXCLK, next_mii_rx_dv, 1'b0, ~reset_n, ~mii_rx_clk, mii_rx_dv);
greg #(1) mii_rx_er_r1(phy_RXCLK, next_mii_rx_er, 1'b0, ~reset_n, ~mii_rx_clk, mii_rx_er);

wire [7:0] next_int_rx_din;
wire next_int_rx_dv;
wire next_int_rx_er;
gmux #(10, 1) rx_mux
  (.d({rx_er_0, rx_dv_0, rx_din_0, mii_rx_er, mii_rx_dv, mii_rx_din}),
   .sel(gmii),
   .z({next_int_rx_er, next_int_rx_dv, next_int_rx_din}));

greg #(8) int_rx_din_reg(phy_RXCLK, next_int_rx_din, 1'b0, ~reset_n, 1'b1, int_rx_din);
greg #(1) int_rx_dv_reg(phy_RXCLK, next_int_rx_dv, 1'b0, ~reset_n, 1'b1, int_rx_dv); 
greg #(1) int_rx_er_reg(phy_RXCLK, next_int_rx_er, 1'b0, ~reset_n, 1'b1, int_rx_er);

endmodule
$
\end{chunk}

register the output of all this craziness to maintain 125 MHz these
registers act as the cross from the $phy_RXCLK$ domain to the
$int_rx_clk$ domain.  TODO $-$ metastability issues.

\end{enumerate}
